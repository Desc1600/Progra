\documentclass{article}
\usepackage[utf8]{inputenc}
\usepackage{amsmath}
\usepackage{amssymb}
%paquetes que usa el doc

\title{Hoja de trabajo \# 3}
\author{Luis Gerardo Morales Salazar \\Carnet: 2018-1364\\ morales181364@unis.edu.gt}
\date{18 de agosto de 2018}

% datos del encabezado o titulo del doc
\usepackage{natbib}
\usepackage{graphicx}


\begin{document}
\maketitle
% respuestas ejercicio 1
\section{Ejercicio \# 1}

\begin{enumerate}
$Sumar:  s(s(s(0))) \oplus s(s(s(s(0))))
\\\\s(s(s(0))) \oplus  s(s(s(s(0))))
\\s(s(s(s(0))) \oplus  s(s(s(0))))
\\s(s(s(s(s(0))) \oplus  s(s(0))))
\\s(s(s(s(s(s(0))) \oplus  (s(0))))
\\s(s(s(s(s(s(s(0\oplus0))))
\\s(s(s(s(s(s(s(0)))))))$

%respuestas ejercicio 2
\\\\ \section{Ejercicio \#2}
Definir la multiplicación para numeros naturales unarios:

\[n\otimes m := \left\{
\begin{array}{l l}
\\0 & \mbox{si } n=0 \\
\\0 & \mbox{si } m=0 \\
\\0 & \mbox{si } m=0,n=0 \\
\\m & \mbox{si } n=1 \\
\\n & \mbox{si } m=1 \\
\\s(i)\oplus(s(i)\otimes j) & \mbox{si } n=s(i)
\\ \end{array}
\right.\]

%Respuestas ejercicio 3
\section{Ejercicio \#3}

%Respuesta 3.1
\item {$s(s(s(0)))\otimes 0$
\\$s(s(s(0))) \otimes 0 = 0$}

%Respuesta 3.2
\item {$s(s(s(0)))\otimes s(0)$}
\\$s(s(s(0))) \otimes s(0) = s(s(s(0))) \oplus (s(s(s(0)))\otimes 0) = s(s(s(0)))$

%Respuesta 3.3
\item {$s(s(s(0)))\otimes s(s(0))$}
\\$s(s(s(0))) \oplus (s(s(s(0))) \otimes s(0))$
\\$s(s(s(0))) \oplus s(s(s(0)))$
\\$s(s(s(s(0))) \oplus s(s(0)))$
\\$s(s(s(s(s(0))) \oplus s(0)))$
\\ $s(s(s(s(s(s(0))) \oplus 0)))$
\\$s(s(s(s(s(s(0 \oplus 0))))))$
\\$s(s(s(s(s(s(0))))))$
\end{enumerate}

%Respuestas ejercicio 4
\section{Ejercicio \#4}
\begin{enumerate}

%Respuesta 4.1
\item{$a\oplus s(s(0))=s(s(a))$} 
\\\\Caso base a=0 
\\$0 \oplus s(s(0)) = s(s(0))$ 
\\$(s(0 \oplus 0)) =s(s(0))$
\\$s(s(0)) = s(s(0))$ 
\\\\Caso inductivo a= s(i) 
\\$s(i) \oplus s(s(0)) = s(s(s(i))) $ 
\\$s(s(i)) \oplus s(0) = s(s(s(i)))$ 
\\$s(s(s(i\oplus 0))) = s(s(s(i)))$ 
\\$s(s(s(i))) = s(s(s(i)))$

%Respuesta 4.2
\item{$a \otimes b = b \otimes a$}
\\\\Caso base a = 0 
\\$0 \otimes b = b \otimes 0$ 
\\$0 = 0$ 
\\\\Caso Inductivo a = s(i)
\\$s(i) \otimes b = b \otimes s(i)$ 
\\$s(i) \oplus (s(i)\otimes b) = (b \otimes s(i)) \oplus s(i)$
\\$s(i) \oplus (s(i)\otimes b) = s(i) \oplus (s(i) \otimes b)$

%Respuesta 4.3
\item{$a \otimes (b \otimes c)=(a\otimes b)\otimes c$}
\\\\Caso Base c= 0
\\$a \otimes (b\otimes 0) = (a \otimes b) \otimes 0$
\\$a \otimes 0  = (ab) \otimes 0$
\\$0=0$
\\\\Caso Inductivo a= s(i)
\\$s(i) \otimes (b \otimes c)=(s(i)\otimes b)\otimes c$
\\$s(i) \oplus (s(i) \otimes (b\otimes c)) = (s(i) \oplus (s(i) \otimes b)) \otimes c$
\\ $s(i) \oplus (s(i) \otimes (b\otimes c)) = s(i) \oplus (s(i) \otimes (b\otimes c))$ 

%Respuesta 4.4
\item $(a \otimes b) \otimes c = (a \otimes c) \oplus (b \otimes c)$
\\\\Caso base: $c = 0$
\\$(a \otimes b) \otimes 0 = (a \otimes 0) \oplus (b \otimes 0)$
\\$(ab) \otimes 0 = (0) \oplus (0)$
\\$0 = 0$
\\\\Método inductivo: $c = n \oplus 1$
\\$(a \otimes b) \otimes (n \oplus 1) = (a \otimes (n + 1)) \oplus (b \otimes (n \oplus 1))$
\\$(a \otimes (n \oplus 1) \oplus (b \otimes (n \oplus 1)) = (an \oplus a) \oplus (bn \oplus b)$
\\$(an \oplus a) \oplus (bn \oplus b) = (an \oplus a) \oplus (bn \oplus b)$
\\$an \ominus an \oplus bn \ominus bn \oplus a \ominus a \oplus b \ominus b = 0$
\\$0 = 0$

% documento creado en sharelatex.com
\end{enumerate}
\end{document}